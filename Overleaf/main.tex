% ****** Start of file apssamp.tex ******
%
%   This file is part of the APS files in the REVTeX 4.1 distribution.
%   Version 4.1r of REVTeX, August 2010
%
%   Copyright (c) 2009, 2010 The American Physical Society.
%
%   See the REVTeX 4 README file for restrictions and more information.
%
% TeX'ing this file requires that you have AMS-LaTeX 2.0 installed
% as well as the rest of the prerequisites for REVTeX 4.1
%
% See the REVTeX 4 README file
% It also requires running BibTeX. The commands are as follows:
%
%  1)  latex apssamp.tex
%  2)  bibtex apssamp
%  3)  latex apssamp.tex
%  4)  latex apssamp.tex
%
\documentclass[%
 reprint,
%superscriptaddress,
%groupedaddress,
%unsortedaddress,
%runinaddress,
%frontmatterverbose, 
%preprint,
%showpacs,preprintnumbers,
%nofootinbib,
%nobibnotes,
%bibnotes,
 amsmath,amssymb,
 aps,nofootinbib
%pra,
%prb,
%rmp,
%prstab,
%prstper,
%floatfix,
]{revtex4-1}
\usepackage{dsfont}
\usepackage{graphicx}
\usepackage{dcolumn}
\usepackage{bm}
\usepackage{color}
\usepackage{epsfig}
\usepackage{hyperref}
\usepackage{natbib}
\usepackage{float}
\usepackage{footmisc}
\definecolor{lightred}{rgb}{1,0.5,0.5}
\definecolor{lightgreen}{rgb}{0.5,1,0.5}
\definecolor{lightblue}{rgb}{0.5,0.5,1}
\definecolor{lightcyan}{rgb}{0.5,0.75,0.75}
\definecolor{lightmagenta}{rgb}{0.75,0.5,0.75}
\definecolor{customgreen}{rgb}{0.494,1,0.502}


\newcommand{\mueV}{\mathinner{\mu\mathrm{eV}}}
\newcommand{\meV}{\mathinner{\mathrm{meV}}}
\newcommand{\eV}{\mathinner{\mathrm{eV}}}
\newcommand{\keV}{\mathinner{\mathrm{keV}}}
\newcommand{\MeV}{\mathinner{\mathrm{MeV}}}
\newcommand{\GeV}{\mathinner{\mathrm{GeV}}}
\newcommand{\TeV}{\mathinner{\mathrm{TeV}}}


%%%%%%%%%%%%%%%%%%%%%%%%%%%%%%%%%%%%%%%%%%%%%%%%%%%%%%%%%%%%%%%%%%%%%%%%
%%%%%%%%%%%%%%%%%%%%%%%%%%%%%%%%%%%%%%%%%%%%%%%%%%%%%%%%%%%%%%%%%%%%%%%%

\begin{document}

\title{Axion-Photon Coupling, Superradiance, and Dark Matter}
\author{Lucas Kang, Anubhav Mathur, and Erwin H. Tanin}
%\date{March 9, 2017}


\maketitle


\tableofcontents


\section{Axion-photon coupling and bounds from black hole superradiance}


A spinning black hole, through a process called \textit{superradiance}, can emit axions which carry away part of its angular momentum. These emitted axions would accumulate and form a cloud around the black hole. Once the cloud exceeds a certain critical density, a Bose-enhancement powered exponential growth would ensue. The axion cloud may then grow so big that it takes away a large, $O(1)$ portion of the black hole's angular momentum. This superradiant instability is sensitive to the axion mass and so bounds can be put on the axion mass by observing the spectrum of black hole spins.

The story might not be that simple. As the axion cloud becomes denser and denser, new effects may enter the picture. If the axion couples strongly enough to photons, one might worry that the axion cloud would deplete due to axion-to-photon conversions before it could grow enough and steal a significant fraction of the black hole's angular momentum, thus removing the bound on axion mass that would otherwise be there. The question is: how does the bound from superradiance change as we vary the strength of axion-photon coupling?








\section{Axion-photon coupling and dark matter overclosure bound}
The same depletion mechanism may rule out the possibility that axion is the dark matter. If axions were the dark matter, they were presumably produced by some mechanism in the early universe at high energies. However, due to their subsequent decay to photons, the axion abundance may get depleted so much that they cannot make up the whole dark matter density today.

We expect the depletion mechanism to work differently for axion cloud in the vicinity of a black hole today and for axion condensate in the early universe. While there is essentially nothing around the black hole in the former case, there should be a dense plasma 








\section{Superradiance and dark matter bounds}
The severity of the bounds coming from black hole superradiance and the viability of axion as the dark matter are linked through the strength of axion-photon coupling.









%\nocite{*}

\newpage
\bibliography{references}
\bibliographystyle{unsrt}
%\bibliographystyle{h-physrev} 



\end{document}